\section{Discussion}
\label{sec:5}

This work studied the effect of train design features on the boarding and alighting time (BAT). The approach was based on similar studies done previously by \cite{Ref26,Ref27,Ref28,Ref29,Ref30}, in which laboratory experiments were performed at the Pedestrian Accessibility Movement Environment Laboratory (PAMELA) in University College London and observation was done using a complete CCTV footage analysis of two weeks (morning and afternoon peak hours) at Green Park Station in London Underground. 

The results of the laboratory experiments showed the importance of the door width, vestibule setback and vertical gap on the BAT. The combination of wider doors (1.80 m), larger vestibule setback (800 mm) and smaller vertical gap (50 mm) presented the lowest tb, reaching 0.65 s/pass on average. Our hypothesis had been that a larger gap size would lead to a lower tb, and this phenomenon was in concordance to our hypothesis. However, in the case of ta the statistical test showed no significant differences (even if larger vertical gaps reached a lower ta). It should be noted that this phenomenon was observed in some runs with a door width of 1.50 m and in many runs with a door with of 1.80 m. It is assumed that, in alighting, a major factor which decides the number of alighting passengers within a given time could be the train’s internal layout, and it has been observed in the experiment runs with a door width of 1.80 m that two parallel streams (or lines of flow) of alighting passengers often (but not always) emerged at the door, while for 1.30 m there was only one stream. In these cases, the impact of the vertical gap can become relatively less important and thus in some cases a larger vertical gap gave a lower ta. In the case of boarding experiments, usually no congestion inside the train occurred (as boarding started when alighting completely or almost finished), and thus the step becomes a factor that determines tb. These results are supported by the MANOVA analysis, in which the vertical gap, the door width and the vestibule setback has an impact on tb (i.e. p-value less than 0.05). 

Similarly to in the laboratory experiments, the results from the observations at Green Park Station can be interpreted as the BAT being influenced by the vertical gap. A small vertical gap of 170 mm could reduce the BAT by up to 13 percent. We thought that this result could also be affected by the types of passengers (e.g. passengers with restricted mobility were more attracted to use Door 3 than other doors over the length of the platform). Nevertheless, in terms of the total passengers that boarded and alighted at Door 3, only 0.5 percent used wheelchairs/prams and 2.8 percent carried luggage. 

It must be noted, however, that the objective of our experimental work is not to recommend the ultimate design features, but to shed light onto the magnitude of changes on BAT as a consequence of variations in the door width, vestibule setback and vertical gap. In addition, values of vertical gaps different to zero may cause inaccessibility for people with permanent or temporary disabilities (e.g. pushchair, trolley bag, or encumbrances). In such cases, some parts of the platform may have special facilities, for instance platform humps, as in the case of Green Park Station. In this sense, to compare the laboratory experiments and obtain an "optimal" design feature or "best scenario", further research is needed. More runs would help to reduce possible errors and differences between results at PAMELA. However, in this study the resources were limited, and therefore between 2 and 3 runs per scenario were performed at PAMELA. In addition, further research is needed to examine if the impulse of passengers is influenced by the interaction between flow size (number of passengers), type of passengers and gap size at the PTI. For instance, it could be interesting to obtain the speed of passengers and compared to some studies such as Weidmann \cite{Ref43}, in which the author obtained relationship between density and speed for different type of passengers (shoppers have a free-flow speed of 1.04 m/s, commuters 1.45 m/s, and tourists 0.99 m/s). For the further studies, we would like to first do field studies and then laboratory experiments to test the same cases or even new situations not seeing at the existing stations. This approach would be recommended to better connect the experiments with the field studies. 

In conclusion, the use of laboratory experiments helped to test different situations ("what if" scenarios) in a controlled environment. This would be difficult to do in a real situation due to the different variables affecting the layout and vehicles of existing public transport systems. In addition, few laboratories such as PAMELA have been built in the world, which has led us to be in a privileged position and be able to perform new research. Currently, new experiments are simulating the use of a waiting area or a "stay clear" to avoid alighting being blocked by passengers waiting in front of the doors. This research will include more observation and experiments to better understand the effect of other variables such as staircases, level of demand and type of passengers on the behaviour of passengers.