\section{Revisión Bibliográfica}
\label{sec:2}

El texto del artículo tiene interlineado sencillo; 12 puntos de tamaño de fuente; tipo de letra Times New Roman; las páginas están numeradas; se utiliza cursiva para las palabras en inglés en lugar de subrayado (excepto en las direcciones URL); y todas las ilustraciones, ecuaciones, figuras y tablas se encuentran enumeradas y colocadas en los lugares del texto apropiados, en vez de al final.
La revisión bibliográfica debe mencionar el marco teórico y la brecha del conocimiento. Para ello se puede hacer uso de citas en revistas científicas, conferencias, libros, capítulos de libros, tesis, entre otros documentos. 
Las citas en el texto deben ir con el apellido del autor y año de publicación (e.g. las variables se relacionan de acuerdo al tiempo de servicio de pasajeros (Tyler, 2002). 
Si son dos autores se debe incluir ambos apellidos y año (e.g. los estudios de Pretty y Russel (1988) demuestran que la variable más relevante es el número de pasajeros). 
En el caso de 3 o más autores, se usa et al (e.g. Fernández et al (2008) ha estudiado los paraderos bajo diferentes configuraciones).
Si es un manual se puede usar las siglas de dicho documento (e.g. TRB, 2010). 
