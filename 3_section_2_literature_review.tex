\section{Revisión Bibliográfica}
\label{sec:2}

El movimiento peatonal se ha estudiado desde hace más de 40 años cuando se comienza a incorporar la evaluación de los espacios de circulación mediante el Nivel de Servicio Peatonal (NSP) de Fruin (1971). El NSP va del Nivel A (flujo libre sin conflictos) al Nivel F (densidad crítica, flujo esporádico, paradas frecuentes y contacto físico), donde el Nivel E es igual a la capacidad. Desde entonces, diferentes autores han estudiado las características de peatones tales como la velocidad, densidad, flujo, entre otros (Banerjee et al, 2018; Vallejo-Borda et al, 2020). Actualmente, este indicador se utiliza en manuales conocidos como el Highway Capacity Manual (HCM, 2000; 2013).

El NSP es un importante indicador para identificar problemas de congestión en andenes, áreas de espera y escaleras. Sin embargo, el problema es que se basa en una visión global o macroscópica, en la que el flujo de peatones se entiende como "dinámica de fluidos" (Still, 2000). Según Evans y Wener (2007), la densidad general utilizada en el este indicador no predice qué espacio presenta más interacción entre los pasajeros. Los autores estudiaron la densidad, el estrés y los desplazamientos en trenes donde los pasajeros tienen que sentarse junto a otros, y encontraron que el nivel de estrés aumentaba a medida que aumentaba la densidad. En este sentido, pareciera ser que utilizar valores promedio de densidad no es adecuado para identificar que parte de la interfaz tren-andén es la más congestionada, especialmente si se implementan medidas de gestión y accesibilidad como puertas en andén o línea amarilla de seguridad.

Respecto a las puertas en andén, algunos autores (Coxon et al, 2010) indican que estos elementos cambian el comportamiento de pasajeros, aumentando los tiempos de subida y bajada. Según los autores, las puertas en andén se limitan al número de puertas de los trenes, a la cantidad de carros del tren y al diseño del andén. Sin embargo, no está claro cómo los autores llegaron a esta conclusión y si hay alguna evidencia que la respalde. Otros autores (Qu y Chow, 2012) estudiaron el uso de estas puertas, las cuales mejoraron la ventilación y la detección de humo en los túneles del metro, sin embargo, el tiempo de evacuación en los andenes aumentó debido a la inconsistencia de la parada del tren en la misma posición en el andén o por la fragilidad de sus materiales.

Por otra parte, en experimentos a escala real de University College London’s Pedestrian Accessibility Movement Environment Laboratory (PAMELA) se identificó que las puertas en andén no tienen un gran impacto en los tiempos de subida y bajada, y solo mejora dichos tiempos en situaciones de congestión, ya que los pasajeros se ubican a los costados de las puertas en vez de al frente de estas (De Ana Rodríguez et al, 2016; Seriani et al, 2017a; Seriani et al, 2017b). Estos estudios fueron corroborados por los mismos autores en observación en estaciones del Metro de Londres, obteniendo resultados similares.

Con respecto a otras medidas, algunos estudios recientes (Prasertsubpakij y Nitivattananon, 2012; Enginöz y Şavlı, 2016) han estudiado la accesibilidad mediante el uso de encuestas a personas con movilidad reducida. En términos de diseño, algunos manuales como el del Metro de Londres (2012) indican que los andenes deben tener un ancho mínimo de 3,0 m, y la línea amarilla al borde del andén es esencial para alertar sobre la proximidad al tren. En el Metro de Londres (2015) se observó cómo variaba el comportamiento de los usuarios con respecto a distintos diseños de la línea amarilla en el borde del andén. El estudio demostró que los tiempos de demoras no se vieron afectados negativamente por un mayor ancho de línea amarilla, y por ende no hubo información sobre un impacto en las operaciones de las estaciones. Además, las observaciones realizadas en terreno demostraron que los pasajeros generalmente están más alejados del borde del andén que antes de la prueba, y cuando menos se respetaba la línea era cuando la gente no alcanzaba a subirse al tren, dejando a los pasajeros esperando el próximo tren en el andén. En ese caso, los pasajeros avanzaban para subir al tren, pero no retrocedían detrás de la línea amarilla si no podían abordar.

Asimismo, en el caso de Metro de Santiago, algunos autores como Amestoy (2015) han estudiado diferentes variables para definir accesibilidad y los tipos de pasajeros en estaciones basadas en los Decretos 142 y 50 amparados en la Ley de Accesibilidad Universal (2010) usados para el diseño de estos espacios. La autora pone énfasis en variables como la línea amarilla de seguridad en el andén, la cual debe incluir pavimento táctil para poder ser detectada por pasajeros con movilidad reducida, sin embargo no identifica el efecto que esto produce en los pasajeros que suben o bajan del tren. Según el mismo estudio (Amestoy, 2015) el servicio SENADIS establece que las estaciones deben tener al menos un 70% de accesibilidad.

A pesar de los avances en investigación, se requieren nuevos estudios para determinar el efecto de estos elementos en el comportamiento de pasajeros, siendo este el principal objetivo de estudio.

