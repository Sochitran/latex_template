\section{Introduction}
\label{sec:intro}

The interface between the vehicle and the platform at stations is considered the zone where the most interactions occur. In the case of metro or rail systems, this space is called the platform train interface (PTI) by Seriani and Fernandez \cite{Ref1}. For example, according to RSSB \cite{Ref2} more than 3 billion interactions take place within the UK national train network each year, during which 21 percent of the safety risks (injuries and fatalities) and 48 percent of the fatality risks to passengers are produced at the PTI zone.   

Interactions are also related to the dwell time, which is the time each vehicle remains stationary at the station when transferring passengers \cite{Ref3}. The dynamic part of dwell time is defined as the boarding and alighting time (BAT), whilst the static part includes the time of opening and closing of doors. The dwell time depends on the number of passengers boarding and alighting, and their flow. The speed of passengers depends on different design variables such as the difference in height and distance between the vehicle and the platform, the number and width of doors, and the layout inside the vehicle. In addition, the speed of passengers is influenced by operational variables such as the fare collection method, the density of passengers on the platform and inside the vehicle, the behaviour of passengers (e.g. interactions), etc. Moreover, the dwell time affects the capacity of stations, delays and queues of vehicles, which in turn impacts on the frequency and regularity of the services, and therefore on the delays caused to passengers at the PTI. 

To reduce interactions at the PTI, various recommendations can be modelled and then compared to design thresholds \cite{Ref4}. One of the most common measures to represent the degree of congestion is the Fruin’s Level of Service or LOS \cite{Ref5}, which categorizes walkways, stairs and queues from a Level A (free flow) to a Level F (over the capacity).  However, the LOS is based on unidirectional flows and average values (e.g. number of passengers divided by the platform area), and it is therefore difficult to identify which part of the PTI is more congested. In addition, few manuals and recommendations have addressed the problem of design of vehicles and stations. As a consequence, the design of the PTI is inadequate. Therefore, the decision making has been based on particular cases or has used the method of “trial and error”.

To show or provide evidence, a line of research has been developed based on laboratory experiments and observations at University College London’s Pedestrian Accessibility Movement and Environment Laboratory (PAMELA).

The main question of this research is how the train design features such as door width, vestibule setback (distance between the train doors and the seats) and vertical gap (height between the platform and the train) affects the passengers’ boarding and alighting time (BAT). At stations, design standards have focused on accessibility \cite{Ref6}; however, is level access the best solution to reduce the BAT? The hypothesis of this research is that a larger gap size would lead to a slower speed, increasing the BAT. The specific objectives are: a) to review the literature related to the design of vehicles and stations, and their effect on BAT; b) to simulate the boarding and alighting process at PAMELA; c) to compare the BAT from the laboratory experiments with London Underground (LU) stations.

In this work, only horizontal and vertical gaps (e.g. less than 300 mm) were studied based on observation in existing metro stations (i.e. urban railway systems). Bigger differences in height and distance between the vehicle and the platform (e.g. between 200 and 600 mm) are considered as further research to represent other railway systems (e.g. commuting or long-distance trains).

This paper is composed of five sections. The next section describes existing studies that have measured the BAT, followed by a section that explains the methods of this work.  The fourth section presents the laboratory and observed results. Finally, a discussion and future work of the laboratory experiments and observations at LU stations is then provided.