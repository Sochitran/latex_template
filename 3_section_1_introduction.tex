\section{Introduction}
\label{sec:intro}
Las estaciones de metro se pueden estudiar en diferentes espacios de circulación peatonal: tren-andén, andén-escaleras, mesanina, espacio complementario (por ejemplo, comercio) y ciudad (nivel calle). Sin embargo, el espacio donde más interacciones se produce es la interfaz tren-andén, donde se realiza la subida y bajada de pasajeros \cite{seriani_fernandez}. Cuando dicha interfaz no posee un diseño adecuado, los pasajeros deben recorrer largas distancias y moverse en espacios inseguros.

En el caso internacional, ejemplos como el Reino Unido muestran que cada año se producen más de 3 mil millones de interacciones en la red de trenes, en donde el 48\% de los riesgos de fatalidad para los pasajeros se producen en dicha interfaz \cite{RSSB}. Por lo tanto, este espacio complejo presenta diferentes riesgos y peligros para los pasajeros. Los accidentes pueden ocurrir durante la subida y bajada o simplemente en el borde del andén cuando los pasajeros esperan la llegada del tren. 

En el caso de Santiago, el informe preparado por Metro de Santiago muestra que entre el año 2017 y el mes de marzo del 2019 se presentaron 54 intentos de suicidios en estaciones, siendo de estos 20 casos fatales y el resto intentos frustrados (TVN, 2019). Una cifra aún más preocupante nos muestra como entre el año 2017 y 2019 estos casos aumentaron en un 39\%. Entre los meses de enero y marzo de 2019, ya se lleva registro de 11 casos. Los suicidios representan un tipo de discapacidad psicológica de las personas (DS 50, 2016).

Para mejorar la seguridad en dicha interfaz, diferentes sistemas de metro han implementado medidas de gestión de pasajeros y accesibilidad en estaciones. Un ejemplo, son las puertas en andén, las cuales evitan que pasajeros caigan a las líneas del tren y permiten identificar donde se ubica cada puerta del tren (Clarke y Poyner, 1994; Kyriakidis et al, 2012). En el caso de Santiago, las nuevas Líneas 3 y 6 cuentan con este tipo de puertas, las cuales tienen un ancho de 2,0 m y se abren al mismo tiempo que las puertas del tren. Cuando no es posible implementar las puertas en andén, se requieren de otras medidas de accesibilidad, las cuales no solo deben favorecer el acceso al transporte público sino también su uso (Tyler, 2002). Por ejemplo, la interfaz tren-andén de la Línea 1 del Metro de Santiago cuenta con una línea amarilla al borde del tren, para evitar que pasajeros se aproximen al borde del andén. Si bien estas medidas (puertas en andén y línea amarilla) se utilizan principalmente por temas de seguridad, se desconoce el efecto que tienen en el comportamiento de pasajeros. En particular, existe una variabilidad de espesores de línea amarilla, lo cual indica la falta de un estándar de seguridad para este espacio.

La accesibilidad en este estudio se plantea como un derecho para todas las personas, el medio para permitir la participación, la autonomía y la vida independiente, evocando un nuevo paradigma de que la discapacidad se centra en la relación de la persona con el entorno, más que su relación funcional. De esta manera, se debe diseñar entornos, más aún, sistemas de transportes, desarrollando tecnologías de accesibilidad e inclusión pensando en la diversidad funcional de los pasajeros (SENADIS, 2016). En Chile, según el Estudio Nacional de Discapacidad (2015), un 16,7\% de la población presentan alguna discapacidad, la cual se subdivide en personas con discapacidad y personas con movilidad reducida.  Bajo el mismo estudio, del total de la población, la accesibilidad para un 13\% es urgente, para un 60\% es necesaria y para el 100\% es confortable. Esto conllevó a que para el año 2018 se implementara la Ley de Accesibilidad, incluyendo normas de accesibilidad enfocadas en el espacio público según el Decreto 50 del Ministerio de Vivienda y Urbanismo (DS50, 2016). Sin embargo, no especifica normas para el transporte subterráneo y, por consiguiente, estaciones subterráneas. Existen diversas medidas para mejorar accesibilidad en el espacio Tren-Andén, específicamente este estudio se enfoca en las puertas de anden y en el uso de la línea amarilla en la interfaz, ya que los pasajeros pueden quedar atrapados, caer en las vías o tropezar a bordo de los trenes, sufriendo lesiones.

Frente a esto, se propone como objetivo de estudio identificar el efecto de las medidas de gestión y la accesibilidad en el comportamiento de pasajeros en estaciones de metro. Para estudiar la interfaz tren-andén se propone observar en terreno estaciones con puertas en andén y la variabilidad del espesor de línea amarilla, para luego realizar experimentos a escala real en el Laboratorio de Dinámica Humana (LDH) de la Universidad de los Andes. Los resultados se podrán transformar en recomendaciones para mejorar los estándares de seguridad de estaciones de metro.

El resto del documento está organizado de la siguiente manera. Primero, se presentan los estudios existentes sobre medidas de gestión de pasajeros y accesibilidad en la interfaz tren-andén. Posteriormente, se describe los métodos utilizados en la observación en terreno y experimentos a escala real. Finalmente se analizan los resultados, para luego proponer recomendaciones y futura investigación.
