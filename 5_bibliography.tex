\begin{thebibliography}{1}
    \bibitem{Ref1}
    Seriani, S., Fernandez, R.: Planning guidelines for metro-bus interchanges by means of a pedestrian microsimulation model in Chile. Transportation Planning \& Technology  \textbf{38(5)}, 569-583 (2015),
    \bibitem{Ref2}
    RSSB: Platform Train Interface Strategy. Rail Safety and Standards Board (2015), London,
    \bibitem{Ref3}
    TRB: Transit Capacity and Quality of Service Manual, 3rd Edition. Transportation Research Board, Washington D.C. (2013),
    \bibitem{Ref4}
    LUL: Station Planning Standards and Guidelines. London Underground Limited, London (2012),
    \bibitem{Ref5}
    Fruin, J.J.: Designing for pedestrians: a level-of-service concept. Highway Research Record  \textbf{377}, 1-15 (1971),
    \bibitem{Ref6}
    Stationery Office: The Rail Vehicle Accessibility Regulations. Description of the gap size is at Regulations 23.1., London (1998),
    \bibitem{Ref7}
    TRB: Highway Capacity Manual 2000, Special Report 209. Transportation Research Board, Washington D.C. (2000),
    \bibitem{Ref8}
    Fernandez, R., del Campo, M.A., Swett, C.: Data collection and calibration of passenger service time models for the Transantiago system. In: Proceedings of the European Transport Conference, (2008), Noordwijkerhout.
    \bibitem{Ref9}
    Tirachini, A.: Bus dwell time: the effect of different fare collection systems, bus floor level and age of passengers. Transportmetrica A: Transport Science  \textbf{9(1)}, 28-49 (2013),
    \bibitem{Ref10}
    Harris, N.G.: Train boarding and alighting rates at high passenger loads. Journal of advanced transportation \textbf{40(3)}, 249-263 (2006),
    \bibitem{Ref11}
    Harris, N.G., Anderson, R.J.: An international comparison of urban rail boarding and alighting rates. In: Proceedings of the Institution of Mechanical Engineers, Part F: Journal of Rail and Rapid Transit \textbf{221(4)}, 521-526 (2007),
    \bibitem{Ref12}
    Weston, J. G.: Train Service Model - Technical Guide, London Underground Operational Research Note, London, 89/18, (1989),
    \bibitem{Ref13}
    Harris, N.G.: Increased Realism in Modelling Public Transport Services. In: Proceedings of the PTRC 22nd European Transport Forum stream H, 1-12 (1994), Warwick, UK,
    \bibitem{Ref14}
    Rudloff, C., Bauer, D., Matyus, T., Seer S.: Mind the gap: boarding and alighting processes using the social force paradigm calibrated on experimental data. In: Proceeding of the 14th International IEEE Conference on Intelligent Transportation Systems, 353-358 (2011),
    \bibitem{Ref15}
    Sun, L., Tirachini, A., Axhausen, K.W., Erath, A., Lee, D.H.: Models of bus boarding and alighting dynamics. Transportation Research Part A \textbf{69}, 447-460 (2014), 
    \bibitem{Ref16}
    Rashidi, S., Ranjitkar, P.: Estimation of bus dwell time using univariate time series models. Journal of Advanced Transportation \textbf{49(1)}, 139-152 (2015), 
    \bibitem{Ref17}
    Qi, X.U., Baohua, M.A.O., Minggao, L.I., Xujie, F.E.N.G.: Simulation of passenger flows on urban rail transit platform based on adaptive agents. Journal of Transportation Systems Engineering and Information Technology \textbf{14(1)}, 28-33 (2014), 
    \bibitem{Ref18}
    Zhang, Q., Han, B., Li, D.: Modeling and simulation of passenger alighting and boarding movement in Beijing metro stations. Transportation Research Part C \textbf{16(5)}, 635-649 (2008), 
    \bibitem{Ref19}
    Davidich, M., Geiss, F., Mayer, H.G., Pfaffinger, A., Royer, C.: Waiting zones for realistic modelling of pedestrian dynamics: A case study using two major German railway stations as examples. Transportation Research Part C \textbf{37}, 210-222 (2013), 
    \bibitem{Ref20}
    Li, D., Daamen, W., Goverde, R.M.: Estimation of train dwell time at short stops based on track occupation event data: A study at a Dutch railway station. Journal of Advanced Transportation, (2016), DOI: 10.1002/atr.1380
    \bibitem{Ref21}
    Wiggenraad, P.B.L.: Alighting and boarding times of passengers at Dutch railway stations - analysis of data collected at 7 stations in October 2000. TRAIL Research School: Delft University of Technology, (2001), Delft, 
    \bibitem{Ref22}
    Harris, N.G., Risan, Ø., Schrader, S.J.: The impact of differing door widths on passenger movement rates. WIT Transactions on The Built Environment  \textbf{155}, 53-63 (2014), 
    \bibitem{Ref23}
    Heinz, W.: Passenger service times on trains-theory, measurements and models. Ph.D. Thesis, Royal Institute of Technology, (2003), Stockholm, 
    \bibitem{Ref24}
    RSSB: Management of on-train crowding Final Report. Rail Safety and Standards Board (2008), London,
    \bibitem{Ref25}
    Fernandez, R., Zegers, P., Weber, G., Tyler, N.: Influence of platform height, door width, and fare collection on bus dwell time. Laboratory evidence for Santiago de Chile. Transportation Research Record \textbf{2143}, 59-66 (2010), 
    \bibitem{Ref26}
    Seriani, S., Fernandez, R.: Pedestrian traffic management of boarding and alighting in metro stations. Transportation Research Part C \textbf{53}, 76-92 (2015), 
    \bibitem{Ref27}
    De Ana Rodriguez, G., Seriani, S., Holloway, C.: Impact of platform edge doors on passengers’ boarding and alighting time and platform behaviour. Transportation Research Record \textbf{2540}, 102-110 (2016), 
    \bibitem{Ref28}
    Seriani, S., Fujiyama, T., Holloway, C.: Exploring the pedestrian level of interaction on platform conflict areas by real-scale laboratory experiments. Transportation Planning \& Technology \textbf{40(1)}, 100-118 (2017), 
    \bibitem{Ref29}
    Seriani, S., Fujiyama, T.: Experimental Study for Estimating the Passenger Space at Metro Stations with Platform Edge Doors.Transportation Research Record, (2018), DOI: 0361198118782027,
    \bibitem{Ref30}
    Holloway, C., Thoreau, R., Roan, T-R., Boampong, D., Clarke, T., Watts, D.: Effect of vertical step height on boarding and alighting time of train passengers. In: Proceedings of the Institution of Mechanical Engineers Part F Journal of Rail and Rapid Transit \textbf{230(4)}, 1234-1241 (2016), 
    \bibitem{Ref31}
    Daamen, W., Lee, Y., Wiggenraad, P.: Boarding and alighting experiments: an overview of the set up and performance and some preliminary results on the gap effects. Transportation Research Record \textbf{2042}, 71-81 (2008), 
    \bibitem{Ref32}
    Atkins: Significant Steps, Research commissioned by UK Department for Transport, (2004), London,
    \bibitem{Ref33}
    Tyler, N., Childs, C., Boampong, D., Fujiyama, T.: Investigating ramp gradients for humps on railway platforms. Municipal Engineer \textbf{168(2)}, 150-160 (2015),    
    \bibitem{Ref34}
    Fernandez, R., Valencia, A., Seriani, S.: On passenger saturation flow in public transport doors. Transportation Research Part A \textbf{78}, 102-112 (2015), 
    \bibitem{Ref35}
    Fujiyama, T., Thoreau, R., Tyler, N.: The effects of the design factors of the train-platform interface on pedestrian flow rates. Pedestrian and Evacuation Dynamics, Springer International Publishing, 102-112 (2015), 
    \bibitem{Ref36}
    Karekla, X., Tyler, N.: Reduced dwell times resulting from train–platform improvements: the costs and benefits of improving passenger accessibility to metro trains. Transportation Planning and Technology \textbf{35(5)}, 525-543 (2012), 
    \bibitem{Ref37}
    Kretz, T., Grünebohm, A., Schreckenberg, M.: Experimental study of pedestrian flow through a bottleneck. Journal of Statistical Mechanics: Theory and Experiment \textbf{10}, P10014 (2006),
    \bibitem{Ref38}
    Hoogendoorn, S. P. Daamen, W.: Pedestrian behaviour at bottlenecks. Transportation Science \textbf{39}, 147-159 (2005),
    \bibitem{Ref39}
    Seyfried, A., Rupprecht, T., Passon, O., Steffen, B., Klingsch, W., Boltes, M.: New insights into pedestrian flow through bottlenecks. Transportation Science \textbf{43(3)}, 395-406 (2009),
    \bibitem{Ref40}
    Adrian J., Boltes M., Holl S., Sieben A., Seyfried A.: Crowding and queuing in entrance scenarios: influence of corridor width in front of bottlenecks. In: Proceedings of the 9th International Conference on Pedestrian and Evacuation Dynamics, (2018), Lund, Sweden,
    \bibitem{Ref41}
    Childs, C., Fujiyama, T., Brown, I., Tyler, N.: Pedestrian Accessibility and Mobility Environment Laboratory. In: Proceedings of the 6th International Conference on Walking in the 21st Century, (2005), Zurich,
    \bibitem{Ref42}
    The Observer: XT software, (2014), Available at: http://www.noldus.com/observer,
    \bibitem{Ref43}
    Weidmann, U.: Transporttechnik der Fussgaenger, ETH, Schriftenreihe Ivt-Berichte 90, (2003), Zuerich. (In German),
\end{thebibliography}