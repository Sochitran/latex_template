\section{Methods}
\label{sec:3}

The methods used in this research were based on one period of observation at Green Park Station of the LU and real-scale laboratory experiments at PAMELA. The main variables used in these methods were selected according to three groups reported by Seriani and Fernandez \cite{Ref1}: physical (i.e. vertical and horizontal gap, width of doors, width and length of platforms), spatial (i.e. number of seats, setback), and operational (i.e. density of passengers, BAT, time for each passenger to board and alight). 

\subsection{Set-up of experiments}
\label{sec:3.1}

In experiments at PAMELA, the laboratory (or the experimental setting) consisted of a mock-up of a vehicle and the relevant portion of the platform in front of the doors. According to Childs et al. \cite{Ref41} the use of laboratory experiments could help to separate the effect of external factors that influence the movement of passengers, such as social interactions, activities and safety constraints. In addition, the laboratory environment is an ideal space to change one variable and keep the rest fixed. Therefore, PAMELA represent an ideal opportunity for researchers to test "what if" scenarios. However, this does not mean that the behaviour of passengers during the experiments is the same as the behaviour of passengers at existing stations. Thus, the experiments help to select the "best scenario", which would then be observed afterwards in existing stations. 

The set-up at PAMELA consisted of a half-carriage mock-up of a train with one double door. The platform was 3.60 m wide and 10.80 m long, whilst the train was 2.50 m wide and 10.00 m long. The doorway width of 1.30 m, 1.50 m and 1.80 m were based on the existing and proposed rolling stock as well as the results of a field study on the existing Thameslink stations (see ). .

In total 120 participants were recruited at PAMELA, in which on average 55 percentage were male and 45 percentage were female, and mostly under 40 years old. Two load conditions were tested: a) 45 alighting and 5 boarding; b) 45 boarding and 5 alighting. A complete sound system was provided in order to make the environment seem more familiar to the participants. The sound simulated the train movements, i.e. included the train arriving, braking, door opening alarm, door closing alarm and departure. In addition, cameras were installed at a height of 4.0 m in the laboratory ceiling. The boarding/alighting did not require any particularly skilful actions but just walking and getting on/off the step. Before the first experiment of each day, we ran a couple of dry runs where participants were asked to get on/off but we did not record. Participants familiarised themselves with the experiment environment within these dry runs. 

To achieve a high frequency, each train would be able to stop at a station for a maximum of 45 seconds. This means that there would be only 27 seconds for doors to be fully open. At the maximum, the 50 passengers are supposed to alight or board at the double door. Therefore, those passengers needed to alight or board within the 27 seconds door-fully-open time.

In total, 68 runs, representing train arrival, dwell and departure, were completed at PAMELA. Three door widths were simulated: 1.30 m, 1.50 m, and 1.80 m. The vestibule setback changed from 0 mm to 400 mm, and from 400 mm to 800 mm. In total, 27 scenarios were performed. In this experiment, the vestibule setback is defined as the distance between the train doors and the seats, which is also known as the standback space. These scenarios were repeated for three vertical gaps: 50 mm, 165 mm, and 250 mm. The experiments were always performed in the same order, increasing the step size from one to the next experiment. In all cases the horizontal gap was 275 mm. 

The experiment at PAMELA lasted three days. The first day all the experiments were simulated with the step height of 50 mm, the second day 165 mm and the third day 250 mm. Some people participated in all three days while others in only one or two. It might be possible that in the third day people were familiar with the step height of 250 mm, but we think this is not a major factor because such familiarity would have more impact on experiments within the same day. 

Each session consisted of several runs representing the boarding and alighting process involved with a single train at the station (some sessions included 4 runs, others included 6 runs). In runs 1, 3 and 5, the number of the alighting participants was 45 whereas that of the boarding participants was 5. In runs 2, 4 and 6, the number of the alighting participants was 5 whereas that of the boarding participants was 45. Before runs 1, 3 or 5, the train was loaded so that it was full of participants. For each run it was decided to maintain the same number of passengers on the loaded train so it was designed that 70 participants would be on the train before runs 1, 3 or 5 started and after runs 2, 4, 6. 

In order to keep the platform density the same for each run, experiment organisers asked the participants on the platform to crowd together to maintain the density. The set density varied slightly according to the experiment day because the number of participants was slightly different across the experiment days. Each participant was given a unique ID for each day. This was used to instruct participants in terms of boarding and alighting. Because the number of participants was around 120 whereas the number of movements (alighting/boarding) was 50 in each run, some participants needed to stay where they were. However, all the participants had the same number of alighting/boarding in each session. Participants were unaware until the instructions were given whether or not they would be required to board or alight. 

The basic procedure of each run started with the sound system (e.g. train approaching). Then, an experimenter announced was made to mention who should alight/board when the door opened by means of announcing the boarding and alighting participant IDs. After this announcement, the run was started. Except in the cases where we allowed the doors to remain open until the last passenger movement had taken place, we opened/closed the door with the fully opening duration being 27 seconds. After closing the door, and allowing the participants to settle in the new arrangement, we opened the door again in order for any non-completed passenger movements to be completed. This ensured that the right number of participants were in the right place for the succeeding run and provided observers with a quick check of ongoing performance.

The experiment was recorded by eight video cameras set up at various points on the ceiling of PAMELA and one was set up at ground level to view participant performance over the horizontal gap. Each ceiling-mounted camera had a viewing area of 3.2 m by 4.0 m at the floor level, so that distortion was as small as possible. In addition a manual count was taken by observers of the camera outputs to ensure that the correct numbers of passenger movements were included.

The average boarding time per passenger (tb) was obtained as the ratio between the total boarding time (Tb) and the total number of boarding passengers (Pb) each time the train arrived. Tb is defined as the difference in time between the last passenger boarding and first passenger boarding. For example, if only 35 passengers could board the train, in which the first passenger board the train in t = 1 s and the last passenger board the train in t = 27 s (time when doors are closed), then Tb = 26 s and tb = 0.74 s per passenger. The same calculation was done for the average alighting time (ta = Ta/Pa). In this case Ta is obtained by the difference in time between the last passenger alighting and the first passenger alighting. In terms of load conditions, tb is obtained for the case when 45 passengers are boarding and 5 passengers are alighting. Similarly, ta is calculated in the situation when 45 passengers are alighting and 5 passengers are boarding. 

To compare the mean between samples at PAMELA, a MANOVA was performed, taking into consideration that the door width, vestibule setback and vertical gap were changed. For the statistical test it was used a significance level of 0.05 and the null hypothesis (H0) was that the door width, vestibule setback and vertical gap will have no significant effect on the average alighting time (ta) or average boarding time (tb). This non-parametric test was used considering the small sample size and assuming that the data is not normally distributed.

\subsection{Observations in existing stations}
\label{sec:3.2}

The results of the laboratory experiments at PAMELA were then compared with a complete CCTV footage analysis at Green Park Station on the Jubilee Line during the morning and afternoon peak hours: 8:15 am to 9:15 am and 5:15 pm to 6:15 pm. During these time periods the train frequency was around 30 trains per hour (2 min headway on average). In total, two weeks of videos were observed with the software Observer XT 11 \cite{Ref42}. 

At Green Park Station three double doors were observed. The first door was subject to a higher demand as it was located in front of an exit gate on the platform, whilst at the second door passengers needed to walk along the platform to reach the exit gates. In both doors the vertical gap was equal to 170 mm and the horizontal gap was 90 mm, which was within the range of the laboratory simulations at HDL and PAMELA. The third door had a vertical gap of 0 mm as a platform hump had been installed to produce level access for passengers (see ). This hump had a total length of 27.00 m and the same width as the platform (3.00 m). Therefore, it covered four train doors (two double doors and two single doors) in the second and third carriages of the train. The double doors at the trains in Green Park Station were 1.60 m wide.

Similar to the experiments at PAMELA, to record the BAT in LU observations, the number of passengers boarding (Pb) and alighting (Pa) was counted every five seconds. The counting period was between the time when the doors started opening and the time when the doors were completely closed. 

However, in the observations at Green Park Station, the Ta and Tb were combined, obtaining a BAT of 5 second slices. Similar to previous studies at PAMELA and London Underground \cite{Ref27,Ref28,Ref29} the time slices were used as a corrected metric of the BAT because, as opposed to the conditions in a controlled laboratory experiment, in existing stations a corrected metric is needed to isolate the BAT from external factors such as operational delays due to signal failures or congestion down the line, and stems from the impossibility of controlling the boarding and alighting processes under actual operation. In addition, Pb and Pa at Green Park Station were corrected to eliminate those 5 second time slices in which “late runners” were recorded (i.e. passengers boarding the train after the main group had already boarded). A criterion for precise observations was that those passengers who boarded the train after two or more time slices in which no passengers were observed, were not considered in the BAT. 


To further explore the differences in the boarding and alighting process, the average boarding and alighting profiles were analysed. In order to get results that were directly comparable, relative profiles have been used, which isolate the shape of the curve from the demand. Thus, the relative profiles for each observation are obtained by dividing the number of boardings (alightings) in each 5 seconds interval by the total number of boardings (alightings) in that boarding (alighting) process. The profiles presented are formed by averaging over all observations for each interval. Therefore, they represent the average proportion of boardings (alightings) in any given interval.

For the LU observations, only descriptive statistics were provided. The BAT at Green Park Station was obtained as an expanded study of \cite{Ref27,Ref28,Ref29}, in which the authors stated that the data did not satisfy the requirements for parametric tests (e.g. ANOVA) or even non-parametric tests (e.g. Mann-Whitney). There was no normal distribution and the distribution within each group was not similar. 
