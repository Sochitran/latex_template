\section{Métodos}
\label{sec:3}

El texto del artículo tiene interlineado sencillo; 12 puntos de tamaño de fuente; tipo de letra Times New Roman; las páginas están numeradas; se utiliza cursiva para las palabras en inglés en lugar de subrayado (excepto en las direcciones URL); y todas las ilustraciones, ecuaciones, figuras y tablas se encuentran enumeradas y colocadas en los lugares del texto apropiados, en vez de al final.
En esta sección se debe explicar:
\begin{itemize}
    \item La metodología, la toma de datos, uso de encuestas, modelos, etc. 
    \item También se puede incluir restricciones en el método usado. 
\end{itemize}


Además, se debe definir las variables a utilizar, los escenarios a estudiar (ver \reftabla{tab1}{Tabla 1}), y la forma de comparar dichos escenarios, entre otros.

\begin{table}
  \centering
  \begin{tabular}{cl}
    \toprule
      Brecha vertical (mm) & Experimentos\\
    \midrule
        0 & Caso Base             \\ 
        50 & Parada con andén     \\
        100 & No existente        \\
        150 & Parada normal       \\
        200 & No existente        \\
    \bottomrule
  \end{tabular}
  \caption{Ejemplo de escenarios a utilizar en este estudio}
  \label{tab1} % unique label

\end{table}
