\itemprimerapagina
{Resumen}
    {
        El objetivo de este estudio es identificar el efecto de las medidas de gestión y la accesibilidad en el comportamiento de pasajeros en estaciones de metro. Se comparó dos estaciones de metro que cuentan con puertas en andén y demarcación para indicar dónde están las puertas del tren. Adicionalmente, se estudió la accesibilidad de estaciones de metro. Para esto último se realizaron experimentos en el Laboratorio de Dinámica Humana de la Universidad de los Andes para probar diferentes escenarios de espesores de línea amarilla ubicada al borde del andén. Los resultados indican que estas medidas cambian el comportamiento de pasajeros, y permiten tener una interfaz más segura entre el tren y andén. En relación a los experimentos, un mayor espesor de la línea amarilla es más respetado por parte de los pasajeros, sin embargo si esta línea amarilla tiene pavimento táctil puede ser incómodo o inseguro para usuarios en especial para quienes presentan discapacidad o movilidad reducida. Como futura investigación se sugieren nuevos experimentos y observaciones en estaciones existentes para incorporar otros tipos de usuarios y configuraciones de la interfaz tren-andén.
    }

\itemprimerapagina
{Palabras clave}
    {
        gestión de pasajeros, accesibilidad, interfaz tren-andén, puertas en andén, experimentos, estación de metro.
    }

\itemprimerapaginacursiva
{Abstract}
    {
        The objective of this study is to identify the effect of crowd management and accessibility measures on the behavior of passengers in metro stations. Two subway stations that have platform edge doors and demarcation were compared to indicate where the train doors are. Additionally, the accessibility of metro stations was studied. For the latter, experiments were carried out in the Human Dynamics Laboratory of the Universidad de los Andes to test different scenarios of yellow line located at the edge of the platform. The results indicate that these measures change the behavior of passengers and allow them to move in a safer interface between the train and the platform. In relation to the experiments, a greater width of the yellow line is more respected by the passengers, however, if this yellow line has tactile paving, it can be uncomfortable or unsafe for users, especially those with disabilities or reduced mobility. As future research, new experiments and observations at existing stations are suggested to incorporate other types of users and configurations of the train-platform interface.
    }

\itemprimerapaginacursiva
{Keywords}
    {
        crowd management, accessibility, train-platform interface, platform edge doors, experiments, metro station.
    }