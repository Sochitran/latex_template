\begin{abstract}
    The objective of this work is to study the effect of design features such as door width, vestibule setback and vertical gap on passengers’ boarding and alighting time (BAT) at metro stations. Simulated experiments were performed at University College London’s Pedestrian Accessibility Movement Environment Laboratory (PAMELA). The mock-up included a hall or entrance to the train and a relevant portion of the platform in front of the doors. Different scenarios were tested based on existing stations. Results were compared to observations at Green Park Station of the London Underground (LU). Results from PAMELA showed that wider doors (1.80 m), larger vestibule setback (800 mm) and smaller vertical gap (50 mm) reduced the average boarding time. However, the average alighting time presented no significant differences due to other phenomenon such as congestion or formation of lines of flow at doors. The observation at LU presented a reduction of the BAT when a small vertical gap (170 mm) was presented. More experiments are needed at PAMELA to test the effect of the design features for different densities and types of passengers.
\end{abstract}