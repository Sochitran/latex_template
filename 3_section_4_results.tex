
\section{Results}
\label{sec:4}

\subsection{Experiments at PAMELA}
\label{sec:4.1}

\tref{tab:1}-\tref{tab:6} show the effect of train design features on the average alighting time (ta) and the average boarding time (tb) at PAMELA experiments. 

\begin{table}
  \label{tab:1} % unique label
  \centering
  \begin{tabular}{llll}
    \toprule
    Vestibule setback (mm) & V=50 mm & V=165 mm & V=250 mm \\
    \midrule
        0 & 1.10 & 1.07 & 1.02  \\
        0 & 0.86 & 0.85 & 1.38  \\
		0 & 0.87 & 1.09 &       \\
		0 & 0.95 &      &       \\
		400 & 0.96 & 0.91 & 0.76  \\
		400 & 0.84 & 1.11 & 1.00  \\
		400 &      & 0.99 & 1.01  \\
		800 & 0.79 & 0.92 & 0.83  \\
		800 & 0.73 & 0.85 & 0.91  \\
		800 &      & 0.78 & 1.27  \\
	\bottomrule
  \end{tabular}
  \caption{Average boarding time (tb) in seconds for a door width of 1.30 m at PAMELA}
  
\end{table}

\begin{table}
  \label{tab:2} % unique label\label{tab:1} % unique label
  \centering
  \begin{tabular}{llll}
    \toprule
    Vestibule setback (mm) & V=50 mm & V=165 mm & V=250 mm \\
    \midrule
        0 & 0.78 & 1.43 & 1.88  \\
        0 & 0.92 & 0.91 & 1.18  \\
		0 &      &      & 0.97  \\
		400 & 0.90 & 0.82 & 0.90  \\
		400 & 0.79 & 0.86 & 0.96  \\
		800 & 0.77 & 0.79 & 0.84  \\
		800 & 0.74 & 0.77 & 0.84  \\
	\bottomrule
  \end{tabular}
  \caption{Average boarding time (tb) in seconds for a door width of 1.50 m at PAMELA}
  
\end{table}

\begin{table}
  \centering
  \begin{tabular}{llll}
    \toprule
    Vestibule setback (mm) & V=50 mm & V=165 mm & V=250 mm \\
    \midrule
        0 & 0.72 & 1.12 & 0.79  \\
        0 & 0.76 & 0.93 & 0.76  \\
		0 &      & 0.77 & 1.05  \\
		400 & 0.66 & 0.74 & 0.74  \\
		400 & 0.66 & 0.77 & 0.71  \\
		400 &      & 0.66 & 0.78  \\
		800 & 0.68 & 0.71 & 0.78  \\
		800 & 0.62 & 0.79 & 0.81  \\
		800 &      & 0.71 & 0.91  \\
	\bottomrule
  \end{tabular}
  \caption{Average boarding time (tb) in seconds for a door width of 1.80 m at PAMELA}
  \label{tab:3} % unique label
\end{table}

In the case of the average boarding time (tb) the lowest value is found for a vertical gap of 50 mm with a door width of 1.80 m and a vestibule setback of 800 mm, giving 0.65 s/pass on average (see \tref{tab:1}, \tref{tab:2} and \tref{tab:3}). For the same door width and vestibule setback, if the vertical gap is increased to 165 mm and 250 mm, then tb also increased by 13 percent (0.74 s/pass on average) and 27 percent (0.83 s/pass on average), respectively. 

The MANOVA (with a significance level of 0.05) showed that the three variables (door width, vertical gap and vestibule width) presented significant differences. The null hypothesis (H0) for the statistical test was that the door width, vestibule setback and vertical gap will have no significant effect on the tb. Therefore, it is recommended to have wider doors, larger vestibule setback and smaller vertical gaps to reduce tb. This is also supported by , in which a high number of cumulative passengers are reached for a door width of 1.80 m, a vestibule setback of 800 mm and a vertical gap of 50 mm. In addition, for the case of a door width of 1.80 m the correlation became non-linear above 30 boarders. This could be related to the formation of lines of flows of passengers boarding, which is described in \cite{Ref37,Ref38, Ref39, Ref40}. More experiments should be done to better explain the differences between each scenario. In particular, it could be interesting to understand why in the case of a door width of 1.30 m there appears to be a separation between the 400 mm setback, which is not produced in the other two cases (door width of 1.50 m and 1.80 m).

\begin{table}
  \centering
  \begin{tabular}{llll}
    \toprule
    Vestibule setback (mm) & V=50 mm & V=165 mm & V=250 mm \\
    \midrule
        0 & 1.63 & 2.61 & 2.77  \\
        0 & 1.14 & 1.59 & 1.56  \\
		0 & 2.40 & 1.30 &       \\
		0 & 1.78 &      &       \\
		400 & 0.98 & 1.27 & 1.41  \\
		400 & 1.18 & 1.41 & 1.24  \\
		400 &      & 1.00 & 1.22  \\
		800 & 1.41 & 1.50 & 2.19  \\
		800 & 1.08 & 1.33 & 1.05  \\
		800 &      & 1.22 & 0.90  \\
	\bottomrule
  \end{tabular}
  \caption{Average alighting time (ta) in seconds for a door width of 1.30 m at PAMELA}
  \label{tab:4} % unique label
\end{table}

\begin{table}
  \centering
  \begin{tabular}{llll}
    \toprule
    Vestibule setback (mm) & V=50 mm & V=165 mm & V=250 mm \\
    \midrule
        0 & 1.04 & 1.59 & 1.34  \\
        0 & 1.40 & 1.49 & 1.38  \\
		0 &      &      & 1.58  \\
		400 & 1.26 & 1.49 & 1.25  \\
		400 & 1.40 & 1.09 & 1.46  \\
		800 & 1.07 & 1.34 & 1.38  \\
		800 & 1.12 & 1.36 & 1.31  \\
	\bottomrule
  \end{tabular}
  \caption{Average alighting time (ta) in seconds for a door width of 1.50 m at PAMELA}
  \label{tab:5} % unique label
\end{table}

\begin{table}
  \centering
  \begin{tabular}{llll}
    \toprule
    Vestibule setback (mm) & V=50 mm & V=165 mm & V=250 mm \\
    \midrule
        0 & 0.95 & 1.02 & 0.90  \\
        0 & 1.35 & 1.95 & 0.91  \\
		0 &      & 1.00 & 0.85  \\
		400 & 1.10 & 0.95 & 1.25  \\
		400 & 1.12 & 1.30 & 1.46  \\
		400 &      & 0.81 & 1.00  \\
		800 & 0.90 & 0.82 & 0.92  \\
		800 & 0.93 & 1.20 & 1.09  \\
		800 &      & 0.94 & 0.73  \\
	\bottomrule
  \end{tabular}
  \caption{Average alighting time (ta) in seconds for a door width of 1.80 m at PAMELA}
  \label{tab:6} % unique label
\end{table}

In the case of ta (see \tref{tab:4}, \tref{tab:5} and \tref{tab:6}), the best layout for the lowest ta is represented by a vertical gap of 250 mm with a door width of 1.80 m and a vestibule setback of 0 mm, giving 0.89 s/pass on average. 

However, the MANOVA results (with a significance level of 0.05), showed no significant differences for the vertical gap (p-value higher than 0.05). The null hypothesis (H0) for the statistical test was that the door width, vestibule setback and vertical gap will have no significant effect on ta. Possible causes are due to other phenomenon such as congestion inside the train and formation of lines of flow to alight, which were out of the scope of this study. 

Although the vertical gap presented no significant differences, the door width and vestibule setback presented a p-value lower than 0.05. The ta is reduced up to 60 percent on average when increasing the door width and vestibule setback. Therefore, it is recommended to have wider doors (1.80 m) and larger vestibule setback (800 mm) to reduce ta.

\subsection{Observation at Green Park Station}
\label{sec:4.2}

In the case of the LU observations, the ratio (R) of passengers boarding to those alighting was obtained at Green Park Station for the total video recordings at each door. Door 1 and Door 2 (both with a vertical gap of 170 mm) presented an average value of R equal to 3.4 and 3.8, respectively. However, in the case of Door 3 (level access) the ratio R gave 1.8 on average, i.e. Door 3 presented a value of R half that of the other doors. Because of the similarities in R and vertical gap between Door 1 and Door 2, the boarding and alighting time (BAT) was calculated as an average between both doors (henceforth termed Door 1\&2).

Two types of codes were used with the software Observer (to stablish the time and to register an event) and 6 types of events were processed (train arrival, first passenger enters PTI, door opening, boarding or alighting, last passenger exits PTI, door closing), in which the period of analysis was between the times of the doors being opened and closed. The PTI was defined in consultation with Transport for London as the space between the yellow line on the platform edge and the train doors.

 shows the average boarding and alighting profiles for the selected doors at Green Park Station. In all three cases passengers get off first and then other passengers get on. The alighting process started at 0 s and finished almost at the third time slice (10th - 15th s), whilst boarding started at the second time slice (5th - 10th s) and ended almost at the fifth time slice (20th - 25th s). Door 1\&2 (vertical gap 170 mm) presented a slightly lower cumulative boarding profile compared to Door 3. However, the cumulative boarding profiles tend to compensate their differences and converge to zero at 22.5 s, finishing the process at 32.5 s. In relation to the alighting profile there were no marked differences between the three doors.
 
 The profiles at Green Park Station were also influenced by the total number of passengers boarding and alighting. Therefore, to identify the effect of a vertical gap on the BAT, the demand was classified into three categories for each door: a) 0 to 15 passengers; b) 15 to 25 passengers; c) more than 25 passengers.  shows that the BAT increased as the number of passengers boarding and alighting went up. However, the BAT was also influenced by the vertical gap. Door 1\&2 (vertical gap of 170 mm) presented between 5 and 13 percent lower BAT than Door 3 (level access). The minimum difference was reached in the category more than 25 passengers, reaching a difference of 1.6 s, while the maximum difference was obtained in the category 15 to 25 passengers, reaching a difference of 2.4 s. Therefore, it seems that level access is not always the best scenario to reduce the BAT. A possible explanation would be that in presence of a small vertical gap passengers need to do an impulse to board or alight, and therefore their speed increases, reducing the BAT. Further experiments at PAMELA are needed to measure the impulse of passengers, and therefore verify this behaviour. 

In this study it was not possible to identify the effect of on-board passengers on the BAT. The demand on arrival could not be derived from the variables observed from the videos. Instead, the network management information system at Transport for London provides a level of demand, which only says if this demand is low, medium or high. Therefore, this study is limited only to the analysis of the BAT and the boarding and alighting passengers. To obtain an exact value of passengers on-board and calibrate the system further research is needed.

The results of the observation at Green Park Station should be treated carefully as there could be other factors affecting the behaviour of passengers, which were out of the scope of this study such as the location of staircases, level of demand and type of passengers.
